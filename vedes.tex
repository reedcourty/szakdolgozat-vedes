\documentclass{beamer}
\usetheme{default} %{pittsburgh}
\usecolortheme{albatross}

\setbeamercolor{normal text}{fg=white}
\setbeamertemplate{navigation symbols}{} % Navigációs ikonok off

\usepackage[T1]{fontenc}
\usepackage[utf8]{inputenc}
\usepackage[english,magyar]{babel}

\usepackage{hyperref}

\usebackgroundtemplate{
\includegraphics[width=\paperwidth, height=\paperheight]{background.jpg}
}

% Néhány konstans deklarációja:
\newcommand{\vikszerzo}{Nádudvari György}
\newcommand{\vikszerzomail}{ulqp9p@gmail.com}
\newcommand{\vikkonzulens}{Huszerl Gábor}
\newcommand{\vikcim}{Oktatástámogató rendszerek kiszolgáló infrastruktúrájának felügyeleti lehetőségei}
\newcommand{\viktanszek}{Méréstechnika és Információs Rendszerek Tanszék}
\newcommand{\vikdoktipus}{Szakdolgozat}

\newcommand{\setfootline}[1]{\setbeamertemplate{footline}{\setbeamercolor{footline}{fg=white}\begin{beamercolorbox}[sep=1cm,wd=\textwidth,ht=1cm,left]{footline}{#1}\end{beamercolorbox}}}

\hypersetup{
    bookmarks=true,            % show bookmarks bar?
    unicode=true ,             % non-Latin characters in Acrobat’s bookmarks
    pdftitle={\vikcim},        % title
    pdfauthor={\vikszerzo},    % author
    pdfsubject={\vikdoktipus}, % subject of the document
    pdfcreator={\vikszerzo},   % creator of the document
    pdfnewwindow=true,         % links in new window
    colorlinks=true,           % false: boxed links; true: colored links
    linkcolor=black,           % color of internal links
    citecolor=black,           % color of links to bibliography
    filecolor=black,           % color of file links
    urlcolor=black             % color of external links
}

\title{\vikcim}
\author{\vikszerzo \\ \footnotesize{\texttt{\vikszerzomail}} \\[0.5cm] \normalsize{Konzulens: \vikkonzulens}}
\date{2012. június 14.}

\begin{document}

\section{\vikcim}
\begin{frame}[plain]
\titlepage
\end{frame}

\section{Miről is lesz szó?}
\setfootline{Interstellar Overdrive}
\begin{frame}[t]
\frametitle{Miről is lesz szó?}
\begin{itemize}
    \item Tanulásmenedzsment rendszerek
        \begin{itemize}
            \item fogalma, feladatai
            \item átjárhatóság közöttük és miért fontos ez
        \end{itemize}
    \item Tanulásmenedzsment rendszerek IT infrastruktúrája
        \begin{itemize}
            \item A három rétegű architektúra a Moodle rendszeren bemutatva
        \end{itemize}
    \item Tanulásmenedzsment rendszerek erőforrás igényei
        \begin{itemize}
            \item Alapvető igények
            \item Modellek
        \end{itemize}
    \item Információs technológiai infrastruktúrák
        \begin{itemize}
            \item A klasszikus IT infrastruktúra
            \item Felhőalapú infrastruktúrák
        \end{itemize}
    \item IT infrastruktúrák proaktív menedzsmentje általános és oktatástámogató rendszerek esetén
    \item Összefoglalás
\end{itemize}
\end{frame}

\section{Tanulásmenedzsment rendszerek}

\subsection*{Fogalma}
\setfootline{Remember a Day}
\begin{frame}
\frametitle{Mit nevezünk tanulásmenedzsment rendszernek?}
Tanulásmenedzsment rendszernek, angolul Learning Management Systemnek (LMS-nek), vagy oktatástámogató rendszernek nevezzük azt a szoftver alkalmazást, amely automatizálja az oktatás adminisztrációját, követését, az online kurzusok és az azokkal kapcsolatos események, anyagok kezelését.
\end{frame}

\subsection{Feladata}
\setfootline{@@@@@@@@@@}
\begin{frame}
\frametitle{Mik a tanulásmenedzsment rendszerek feladatai?}
Egy robusztus LMS-nek képesnek kell lennie 
\begin{itemize}
\item központosított és automatizált adminisztrációra,
\item önkiszolgáló és önálló irányítású szolgáltatások nyújtására,
\item oktatási anyagok gyors összeállítására és elérhetőségének biztosítására,
\item konszolidált képzési kezdeményezésekre skálázható, web alapú platformon,
\item a portabilitás és a szabványok támogatására,
\item személyre szabott tartalom előállítására és a tudás újrafelhasználásának lehetővé tételére.
\end{itemize}
\end{frame}

\end{document}